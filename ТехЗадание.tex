\newsection
\section{Техническое задание}
\subsection{Основание для разработки}

Основанием для разработки веб-платформа для анализа и визуализации статистических данных киберспортивной игры CS2 является задание на выпускную квалификационную работу приказ ректора ЮЗГУ от <<  >>     2024 года № 0000-0 <<Об утверждении тем выпускных квалификационных работ и руководителей выпускных квалификационных работ>>.

\subsection{Цель и назначение разработки}

Функциональное назначение разрабатываемой веб-платформы заключается в предоставлении игрокам и киберспортивным тренерам эффективного инструмента отображения статистических данных из матчей с целью повышения эффективности команд и игроков, а так же отображение и предоставление прогнозов на будущие исходы матчей.

Задачами данной разработки являются:
%\begin{itemize}
%\item создание сервиса обработки .dem записей;
%\item создание сервиса загрузки .dem записей;
%\item создание сервиса прогнозирования матчей;
%\item создание базы данных всех матчей и турниров начиная с 2012 года;
%\item реализация backend части веб-платформы микросервисной архитектурой;
%\item реализация frontend части веб-платформы;
%\item создание удобного поиска по платформе.
%\item реализация регистрации и авторизации пользователя.
%\end{itemize}

\subsection{Требования к веб-платформе}

\subsubsection{Требования к данным веб-платформы}

\subsubsection{Функциональные требования к веб-платформе}

\subsubsection{Требования пользователя к интерфейсу веб-платформы}

\subsection{Моделирование вариантов использования}

\subsection{Нефункциональные требования к программной системе}

\subsubsection{Требования к надежности}

\subsubsection{Требования к программному обеспечению}

\subsubsection{Требования к аппаратному обеспечению}

\subsubsection{Требования к оформлению документации}

Разработка программной документации и программного изделия должна производиться согласно ГОСТ 19.102-77 и ГОСТ 34.601-90. Единая система программной документации.
