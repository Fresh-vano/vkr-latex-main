\newsection
\section{Анализ предметной области}
\subsection{Описание предметной области}

Киберспорт — это соревновательная деятельность, связанная с компьютерными играми, где участники, индивидуальные игроки или команды, соревнуются друг с другом в специально организованных матчах и турнирах. Он объединяет технологии, спорт и развлечения, предлагая уникальное сочетание физического мастерства и интеллектуальных способностей. Киберспорт, или электронный спорт, в последние десятилетия превратился из ниши видеоигр в полноценную глобальную индустрию, объединяющую миллионы игроков и зрителей по всему миру. Начавшись как любительские соревнования по видеоиграм в 1996 году, киберспорт быстро набирал популярность, превратившись в организованные и профессионально управляемые международные турниры с крупными призовыми фондами.

В России киберспорт получил официальное признание как вид спорта в 2001 году, а в 2016 году была создана Федерация Компьютерного Спорта России. Это признание подчеркивает значимость и влияние киберспортивной индустрии не только как средства развлечения, но и как сектора, способного способствовать развитию технологий, маркетинга и массовой культуры в стране.

Рынок киберспорта демонстрирует впечатляющий рост. По данным различных аналитических агентств, глобальный рынок киберспорта в 2022 году оценивается в 1.4 миллиарда долларов, а количество его зрителей и фанатов за последний год превысило 500 миллионов человек. Это не только свидетельствует о популярности киберспорта, но и подчеркивает его потенциал с точки зрения рекламы и коммерческой выгоды.

Одним из наиболее знаковых аспектов киберспорта являются его масштабные турниры, которые привлекают внимание миллионов зрителей по всему миру. Турниры такие как The International по Dota 2, Чемпионат мира по League of Legends, и Major по Counter-Strike 2 не только собирают лучших игроков со всего мира, но и предлагают зрелищные шоу, поддерживаемые крупными спонсорами и медиа платформами. Эти события стали важной частью культуры современного интернета, собирая за просмотром миллионы человек одновременно.

Особое внимание заслуживает международный турнир "Игры будущего", который пройдет в Казани в марте 2024 года. Этот масштабный турнир будет включать соревнования по 16 дисциплинам, в том числе CS2, Dota 2, League of Legends и многие другие. Событие обещает стать одним из самых значительных в истории российского киберспорта, собирая лучших игроков со всего мира и предлагая зрителям незабываемое зрелище.
\subsection{Классификация киберспортивных дисциплин}

Киберспортивные дисциплины могут быть классифицированы по жанрам игр, каждый из которых предлагает свои уникальные стратегии, навыки и форматы соревнований. Основные классы включают:

\begin{enumerate}
	\item \textbf{Стратегии в реальном времени (RTS)}: 
	Примеры игр включают StarCraft II и Warcraft III. Этот жанр требует быстрого стратегического планирования, управления ресурсами и армиями.
	
	\item \textbf{Многопользовательские онлайн-боевые арены (MOBA)}: 
	Ключевые игры - Dota 2 и League of Legends. Жанр акцентирует внимание на командных боях, тактическом взаимодействии и индивидуальных навыках игроков.
	
	\item \textbf{Шутеры от первого лица (FPS)}: 
	В эту категорию входят такие игры, как Counter-Strike 2 и Call of Duty, где фокусируется внимание на стрельбе, рефлексах, координации команды и стратегическом планировании.
	
	\item \textbf{Спортивные симуляторы}: 
	Примеры игр - FIFA и NBA 2K. Эти игры имитируют реальные спортивные состязания и требуют знания спортивных правил реальных игр и тактик.
	
	\item \textbf{Боевики и файтинги}: 
	Примеры включают Street Fighter и Super Smash Bros. Жанр сосредоточен на одиночных поединках и требует высокой точности и быстрой реакции.
	
	\item \textbf{Карточные игры}: 
	Игры, такие как Hearthstone и Magic: The Gathering Arena, включают элементы стратегического мышления, управления ресурсами и предвидения действий противника.
\end{enumerate}
\subsection{Компьютерная игра Counter Strike 2}

Counter-Strike 2 (CS2) представляет собой обновление популярной игры Counter-Strike: Global Offensive (CS:GO), разработанное и представленное компанией Valve в 2012 году. Обновление CS2 было выпущено в сентябре 2023 года, принеся значительные улучшения и изменения, включая переход со старого движка Source на новый мощный игровой движок Source 2.Переход на движок Source 2 позволил улучшить графику и производительность игры, а также расширил возможности аналитики и визуализации данных.

CS2 привлекает миллионы игроков по всему миру и является первой  2 среди самых популярных игр в киберспортивной индустрии. По различным оценкам, после обновления игры на новый движок, средний онлайн игры вырос до 800 тысяч активных игроков, а крупные турниры, такие как PGL Major Stockholm 2021, привлекают до 2.748 миллионов зрителей одновременно за просмотром турнира.

Одним из ключевых аспектов выбора CS2 основной дисциплиной для разработки является сохранение записей игр в формате .dem файлов, содержащих подробную информацию о каждом игровом событии, включая движения, стрельбу, покупку и использование предметов игроками, что делает их ценным ресурсом для анализа и улучшения игровых навыков. Аналитики и тренеры используют .dem файлы для разбора игровых моментов, стратегий и тактик, применяемых командами и отдельными игроками. Эти файлы позволяют детально проанализировать игровой процесс, выявить ошибки и моменты для улучшения для каждого отдельного игрока.

Основные правила соревновательной игры CS2 следующие: 
\begin{enumerate}
	\item \textbf{Общая структура}:
	Игра состоит из раундов. В стандартном соревновательном режиме проводится 24 раунда, с командами, играющими 12 раундов за каждую сторону (террористы и контртеррористы).
	
	\item \textbf{Цели раунда}:
	Команда террористов должна либо установить бомбу на одном из мест закладки бомбы и защитить её до взрыва, либо уничтожить всех членов команды контртеррористов. Команда контртеррористов должна либо предотвратить установку бомбы, либо обезвредить её после установки, либо уничтожить всех террористов.
	
	\item \textbf{Покупка оружия и снаряжения}:
	В начале каждого раунда игроки имеют время для покупки оружия и снаряжения. Деньги для покупок зарабатываются в предыдущих раундах посредством убийства противников и победы/проигрыша в раундах.
	
	\item \textbf{Победа в матче}:
	Победа достигается одной из команд, которая первой выигрывает 13 раундов. В случае ничьей (12-12), проводятся дополнительные время.
	
	\item \textbf{Дополнительное время}:
	В дополнительное время команды играют дополнительные раунды, чтобы определить победителя. Это 6 раундов (3 раунда за каждую сторону). Команда, которая выигрывает 4 из этих 6 раундов, становится победителем. В случае ничьи (3-3) проводится еще одно дополнительное время. Дополнительное время добавляется до момента выявления победителя.
	
	\item \textbf{Ограничения по времени и бомбе}:
	Каждый раунд имеет ограничение по времени (1 минута 55 секунд). Если террористы не установили бомбу до истечения времени, побеждают контртеррористы. После установки бомбы, у контртеррористов есть 40 секунд на её обезвреживание.
	
	\item \textbf{Тактические паузы}:
	Команды могут проводить тактические паузы, на которых с командой может общаться их тренер. Во время тактических пауз проводится разбор ошибок и тактик, которые можно применить во время игры. Каждой команде на 1 матч доступно 4 тактические паузы. Каждая тактическая пауза длится 30 секунд.
\end{enumerate}

Эти метрики играют ключевую роль в анализе игровых стратегий и формировании тактик, помогая командам и игрокам улучшать свои результаты и эффективность на киберспортивной арене.
\subsection{Сбор, хранение и анализ статистических данных игры Counter Strike 2}

В .dem файлах игры CS2 содержится информация о различных игровых событиях. Эти события предоставляют ценные данные для анализа стратегий и поведения игроков во время матча. В CS2 анализируются различные метрики для оценки эффективности игроков и команд, основными из которых являются:

\begin{itemize}
	\item \textbf{Процент попаданий в голову (Headshot)}:
	Отражает процент убийств в голову противников от общего числа убийств данным игроком, что отображает точность наведения игрока на цель.
	\item \textbf{Уровень точности стрельбы (Accuracy)}:
	Отражает процент попаданий по противникам от общего числа сделанных выстрелов.
	\item \textbf{Количество убийств и смертей (K/D Ratio)}:
	Соотношение между числом убийств игрока и количеством его смертей.
	\item \textbf{Использование утилит (Utility Usage)}:
	Анализ эффективности использования гранат, дымов, флэш-бангов и других предметов.
	\item \textbf{Позиционирование и перемещение (Positioning and Movement)}:
	Оценка стратегического расположения на карте и способности к маневрированию в бою.
	\item \textbf{Экономическое управление (Economic Management)}:
	Отслеживание способности к управлению финансами для покупки оружия и снаряжения.
\end{itemize}

Существует несколько популярных ресурсов и платформ, предоставляющих статистические данные и аналитику для игры CS2, среди которых основными являются HLTV и Liquipedia:

\begin{enumerate}
	\item \textbf{HLTV}:
	Ведущий сайт в сфере киберспорта для CS2, предоставляющий результаты матчей, их краткую статистику и аналитические статьи. Данный сайт является официальной платформой хранения .dem записей всех матчей, которые были сыграны за все время.
	\item \textbf{Liquipedia}:
	Является обширной вики-платформой по киберспорту, включающая информацию о турнирах, командах и игроках CS2. Она предоставляет статьи о событиях в сфере киберспорта, а также сведения о прошлых и предстоящих проводимых турнирах и матчах.
\end{enumerate}

Дополнительно, веб-сайты, такие как GosuGamers, ESEA или bo3.gg, предлагают статистику и аналитику, связанную с матчами и турнирами CS2. Однако все выше приведенные платформы не предполагают детального изучения и анализа данных игр, а так же прогнозирования будущих исходов матчей.

Развитие технологий анализа данных и их визуализации открывает новые возможности для улучшения тренировочных процессов и стратегического планирования в шутерах от первого лица. Существующие решения, хотя и предоставляют важную информацию, часто ограничены в плане глубины анализа и персонализации данных под конкретные нужды команд или игроков. Разработка специализированной платформы, фокусирующейся на детальном анализе .dem файлов, а так же использовании нейросети для прогнозирования исхода матчей, позволит получить более глубокое понимание игровых процессов, взаимосвязь различных метрик между собой, выявить скрытые паттерны поведения игроков и оптимизировать стратегии команды.

Такая платформа будет включать функции для детального анализа игровых матчей, визуализации статистических данных и создания прогнозов на матчи, что даст командам и игрокам новые инструменты для повышения своего уровня игры. Это не только повысит конкурентоспособность команд на международной арене, но и способствует развитию киберспорта как спортивной дисциплины.